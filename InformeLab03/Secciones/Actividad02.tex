\section{MARCO TEORICO} 
\begin{itemize}
\subsection{Base de datos TSQL:}
	\item SQL (Structured Query Language), Lenguaje Estructurado de Consulta es el lenguaje utilizado para definir, controlar y acceder a los datos almacenados en una base de datos relacional.
Como ejemplos de sistemas gestores de bases de datos que utilizan SQL podemos citar DB2, SQL Server, Oracle, MySql, Sybase, PostgreSQL o Access.
El SQL es un lenguaje universal que se emplea en cualquier sistema gestor de bases de datos relacional. Tiene un estándar definido, a partir del cual cada sistema gestor ha desarrollado su versión propia. 
En SQL Server la versión de SQL que se utiliza se llama TRANSACT-SQL.
EL SQL en principio es un lenguaje orientado únicamente a la definición y al acceso a los datos por lo que no se puede considerar como un lenguaje de programación como tal ya que no incluye funcionalidades como son estructuras condicionales, bucles, formateo de la salida, etc. (aunque veremos que esto está evolucionando).
Se puede ejecutar directamente en modo interactivo, pero también se suele emplear embebido en programas escritos en lenguajes de programación convencionales. En estos programas se mezclan las instrucciones del propio lenguaje (denominado anfitrión) con llamadas a procedimientos de acceso a la base de datos que utilizan el SQL como lenguaje de acceso. Como por ejemplo en Visual Basic, Java, C, PHP .NET, etc.

\subsection{Las instrucciones SQL se clasifican según su propósito en tres grupos:}
	\item El DDL (Data Description Language) Lenguaje de Descripción de Datos..
	\item El DCL (Data Control Language) Lenguaje de Control de Datos.
	\item El DML (Data Manipulation Language) Lenguaje de Manipulación de Datos.

\subsection{Consultas con Pivot :}
	\item Las operaciones con Pivot nos permitirá convertir los resultados de una consulta que se presentan en filas y mostrarlos en columnas.
	\item Pivot utiliza las funciones de agregado para presentar los datos en columnas.
	\item El DML (Data Manipulation Language) Lenguaje de Manipulación de Datos.

\subsection{Grouping Sets:}
	\item GROUP BY GROUPING SETS es una poderosa extensión de la cláusula GROUP BY que permite computar múltiples cláusulas de grupo en una sola declaración. El conjunto de grupos es un conjunto de columnas de dimensión.GRUPO POR CONJUNTOS DE GRUPO es equivalente a la UNIONde dos o más operaciones de GRUPO POR en el mismo conjunto de resultados:
	\item GROUP BY GROUPING SETS((a))es equivalente a la operación de conjunto de agrupación única .GROUP BY a
	\item GROUP BY GROUPING SETS((a),(b))es equivalente a .GROUP BY a UNION ALL GROUP BY b
\end{itemize}





