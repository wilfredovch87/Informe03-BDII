\section{PROCEDIMIENTO} 


\begin{enumerate}[1.]
	\item Escribiendo consultas con el operador PIVOT
	\begin{enumerate}[a)]
	\item Task 1: Escribir una sentencia SELECT para recuperar el numero de clientes para un grupo especifico de clientes.\\
		-  Abrir el SQL Server Management Studio y conectar a la basa de datos (local) usando Windows.\\
		-  Usar la base de datos TSQL\\
		-  Ejecutar el siguiente codigo para crear una vista\\
		\begin{figure}[H]
\begin{center}
\includegraphics[width=8cm]{./Imagenes/1-1}
\end{center}
\end{figure}
		-  Ejecutar la siguiente consulta
\begin{figure}[H]
\begin{center}
\includegraphics[width=4cm]{./Imagenes/1-2}
\end{center}
\end{figure}
		-  Luego modificamos el codigo, aplicando el operador PIVOT.
\begin{figure}[H]
\begin{center}
\includegraphics[width=6cm]{./Imagenes/1-3}
\end{center}
\end{figure}
	\item Task 2: Especifique el elemento de agrupacion para el operador PIVOT. \\
		-  Escribir la siguiente consulta y ejecutar. 
		\begin{figure}[H]
		\begin{center}
		\includegraphics[width=7cm]{./Imagenes/e1-2}
		\end{center}
		\end{figure}
		-  Escribir la siguiente consulta y ejecutar. 
		\begin{figure}[H]
		\begin{center}
		\includegraphics[width=7cm]{./Imagenes/e1-2-1}
		\end{center}
		\end{figure}
		Como se puede observar tiene el mismo resultado que en la consulta de Task1
		-  Modificar la consulta para incluir columnas adicionales desde la vista y ejecutar. 
		\begin{figure}[H]
		\begin{center}
		\includegraphics[width=7cm]{./Imagenes/e1-2-2}
		\end{center}
		\end{figure}
		Como se puede observar tiene el mismo resultado que en la consulta de Task1
	\item Task 3: Use una expresion de tabla común (CTE) para especificar el elemnto de agrupacion para el operador PIVOT.\\
		-  Escribir la siguiente consulta y ejecutar. 
		\begin{figure}[H]
		\begin{center}
		\includegraphics[width=7cm]{./Imagenes/3-1}
		\end{center}
		\end{figure}
		Como se puede observar tiene el mismo resultado que en la consulta de Task1
	\item Task 4: Escribe una instruccion SELECT para recuperar el monto total de ventas para cada cliente y categoria de producto.\\
		-  Escribir la siguiente consulta y ejecutar. 
		\begin{figure}[H]
		\begin{center}
		\includegraphics[width=8cm]{./Imagenes/3-2}\\
		\includegraphics[width=8cm]{./Imagenes/3-3}
		\end{center}
		\end{figure}
	\end{enumerate}

	\item Escribiendo consultas con el operador UNPIVOT
	\begin{enumerate}[a)]
	\item Task 1: Crear y consultar la vista Sale.PivotCustGroups.\\
		-  Escribir la siguiente consulta y ejecutar para crear una vista llamada Sales.PivotCustGroups.\\
		\begin{figure}[H]
		\begin{center}
		\includegraphics[width=8cm]{./Imagenes/e2-1}
		\end{center}
		\end{figure}
		-  Despues escribrir y ejecutar la siguiente consulta.
		\begin{figure}[H]
		\begin{center}
		\includegraphics[width=6cm]{./Imagenes/e1-1-1}
		\end{center}
		\end{figure}
	\item Task 2: Escriba una instruccion SELECT para recuperar una fila para cada pais y grupo de cliente.\\
		-  Escribir la siguiente consulta y ejecutar.\\
		\begin{figure}[H]
		\begin{center}
		\includegraphics[width=8cm]{./Imagenes/e2-2}
		\end{center}
		\end{figure}
	\item Task 3: Eliminar las vistas creadas.
		\begin{figure}[H]
		\begin{center}
		\includegraphics[width=6cm]{./Imagenes/e2-3}
		\end{center}
		\end{figure}
	\end{enumerate}



	\item Escribiendo consultas con las clausulas GROUPING SETS, CUBE, and ROLLUP.
	\begin{enumerate}[a)]
	\item Task 1: Escriba una instruccion SELECT que use LA SUBCLAUSULA GROUPING SETS para devolver el número de
Clientes para diferentes conjuntos de agrupación.\\
		-  Escribir la siguiente consulta y ejecutar. 
		\begin{figure}[H]
		\begin{center}
		\includegraphics[width=7cm]{./Imagenes/e3-1}
		\end{center}
		\end{figure}
	\item Task 2: Escriba una instruccion SELECT que use la subclausula CUBE para recuperar Grouping sets basados en valores de ventas anuales, mensuales y diarios.\\
		-  Escribir la siguiente consulta y ejecutar. 
		\begin{figure}[H]
		\begin{center}
		\includegraphics[width=7cm]{./Imagenes/e3-2}
		\end{center}
		\end{figure}
	\item Task 3: Escriba la misma instruccion SELECT usando la cubclausula ROLLUP.\\
		-  Escribir la siguiente consulta y ejecutar. 
		\begin{figure}[H]
		\begin{center}
		\includegraphics[width=7cm]{./Imagenes/e3-3}
		\end{center}
		\end{figure}
	\item Task 4: Analizar el valor total de ventas por año y mes.\\
		-  Escribir la siguiente consulta y ejecutar. 
		\begin{figure}[H]
		\begin{center}
		\includegraphics[width=7cm]{./Imagenes/e3-4}
		\end{center}
		\end{figure}
	\end{enumerate}
\end{enumerate}



